\chapter{Diseño e Implementación} % Main chapter title
\label{Chapter3} 

Párrafo introductorio.

\section{Diseño del hardware}
\label{sec:hardware}

Bloques constructivos del hardware, se presenta el diagrama de bloques y se describen los mismos

\begin{figure}[ht]
	\centering
	%	\includegraphics[scale=.3]{./Figures/cuadradoAzul.png}
	\caption{Diagrama de bloques del equipo desarrollado}
	\label{fig:DiagramaBloques}
\end{figure}

Selección de los componentes, se presentan los criterios utilizados para en la selección de los componentes.

\section{Prototipo del hardware}
\label{sec:prototipo}

Diseño y construcción de la placa electrónica, se presentan los criterios y el proceso utilizado para el diseño y construcción de la placa electrónica del prototipo.

Montaje del prototipo, se detallan los problemas de construcciones del primer prototipo y se mencionan las correcciones en el diseño de la placa electrónica efectuadas a partir de los problemas de montaje de los componentes.

\begin{figure}[ht]
	\centering
	%	\includegraphics[scale=.3]{./Figures/cuadradoAzul.png}
	\caption{Errores encontrados durante el montaje del primer prototipo}
	\label{fig:ErroresMontaje}
\end{figure}

\section{Diseño del firmware}
\label{sec:firmware}

\subsection{Arquitectura del firmware}

Se presenta la arquitectura seleccionara para el firmware del equipo, se presenta el diagrama de componentes de software del mismo y se describen brevemente las capas del mismo

\begin{figure}[ht]
	\centering
	%	\includegraphics[scale=.3]{./Figures/cuadradoAzul.png}
	\caption{Diagrama de componentes del firmware del equipo}
	\label{fig:DiagramaComponentes}
\end{figure}

\begin{figure}[ht]
	\centering
	%	\includegraphics[scale=.3]{./Figures/cuadradoAzul.png}
	\caption{Diagrama de clases del firmware del equipo}
	\label{fig:DiagramaClases}
\end{figure}

\subsection{Capa de abstracción del hardware}

Se describen las clases que componen la capa de abstracción de hardware

\subsection{Capa de controladores}

Se describen las clases que componen la capa de controladores

\begin{figure}[ht]
	\centering
	%	\includegraphics[scale=.3]{./Figures/cuadradoAzul.png}
	\caption{Diagrama de estado para el control de una puerta, sin sensor de apertura y con liberación electromagnética}
	\label{fig:ControlSinSin}
\end{figure}

\begin{figure}[ht]
	\centering
	%	\includegraphics[scale=.3]{./Figures/cuadradoAzul.png}
	\caption{Diagrama de estado para el control de una puerta, con sensor de apertura y con liberación electromagnética}
	\label{fig:ControlSinCon}
\end{figure}

\begin{figure}[ht]
	\centering
	%	\includegraphics[scale=.3]{./Figures/cuadradoAzul.png}
	\caption{Diagrama de estado para el control de una puerta, sin sensor de apertura y con liberación motorizada}
	\label{fig:ControlConSin}
\end{figure}

\begin{figure}[ht]
	\centering
	%	\includegraphics[scale=.3]{./Figures/cuadradoAzul.png}
	\caption{Diagrama de estado para el control de una puerta, con sensor de apertura y con liberación motorizada}
	\label{fig:ControlConCon}
\end{figure}

\subsection{Tareas del sistema}

Se describen las tareas del sistema operativo de tiempo real y las interacciones entre las mismas.

\begin{figure}[ht]
	\centering
	%	\includegraphics[scale=.3]{./Figures/cuadradoAzul.png}
	\caption{Diagrama de secuencia para la liberación por pulsador de una puerta, sin sensor de apertura y con liberación electromagnética}
	\label{fig:SecuanciaPulsador}
\end{figure}

\begin{figure}[ht]
	\centering
	%	\includegraphics[scale=.3]{./Figures/cuadradoAzul.png}
	\caption{Diagrama de secuencia para la apertura y cierre por tarjeta de proximidad, con sensor de apertura y con liberación electromagnética}
	\label{fig:SecuenciaTarjeta}
\end{figure}

\begin{figure}[ht]
	\centering
	%	\includegraphics[scale=.3]{./Figures/cuadradoAzul.png}
	\caption{Diagrama de secuencia para la activación de alarma por apertura forzada de la puerta}
	\label{fig:SecuenciaForzada}
\end{figure}

\begin{figure}[ht]
	\centering
	%	\includegraphics[scale=.3]{./Figures/cuadradoAzul.png}
	\caption{Diagrama de secuencia para el cambio de configuración del equipo}
	\label{fig:SecuenciaConfiguracion}
\end{figure}

\section{Desarrollo del firmware}
\label{sec:desarrollo}

\subsection{Entorno de desarrollo}

Se describen las herramientas utilizadas para el desarrollo del software

\begin{table}[ht]
	\centering
	\caption{Lista de las herramientas utilizadas para el desarrollo del firmware}
	\begin{tabular}{l c}    
		\toprule
		\textbf{Nombre} 	& \textbf{Característica}	\\
		\midrule
		Elemento 			& Valor	\\
		\bottomrule
		\hline
	\end{tabular}
	\label{tab:HerramientasDesarrollo}
\end{table}

\subsection{Uso del tipo de abstracto de datos}

Se presenta el patrón de programación denominado tipo abstracto de datos y se explica su uso en la implementación del firmware del equipo

\subsection{Pruebas de integración en las tareas del sistema}

Se presenta la problemática de implementar pruebas automatizadas en tareas del sistema operativo FreeRTOS y se explica la herramienta desarrollada para resolver este problema
