% Chapter Template

\chapter{Ensayos y Resultados} % Main chapter title
\label{Chapter4}

Párrafo introductorio.

\section{Pruebas funcionales del hardware}
\label{sec:pruebasHW}

\section{Prototipo del hardware}
\label{sec:PruebasPrototipo}

Mediciones y verificaciones en el prototipo, se detallan las mediciones y ensayos realizados sobre la placa del prototipo para determinar su correcto funcionamiento

Correcciones en el prototipo: Se detallan los errores encontrados en el diseño y las correcciones efectuadas a partir de los ensayos funcionales en la placa del prototipo"

\begin{figure}[ht]
	\centering
	%	\includegraphics[scale=.3]{./Figures/cuadradoAzul.png}
	\caption{Imagen con el error en la definición del encapsulado del reloj de tiempo real}
	\label{fig:ErrorEncapsulado}
\end{figure}

\begin{figure}[ht]
	\centering
	%	\includegraphics[scale=.3]{./Figures/cuadradoAzul.png}
	\caption{Imagen con la corrección efectuada en la placa de prototipo}
	\label{fig:ImagenCorreccion}
\end{figure}

\section{Pruebas en el firmware}
\label{sec:PruebasFirmware}

\subsection{Pruebas unitarias}

Se describen las pruebas unitarias implementadas y los resultados obtenidos

\subsection{Pruebas de integración}

Se describen las pruebas de integración implementadas y los resultados obtenidos

\subsection{Resultados de las pruebas de aceptación del equipo}

Se analizan los resultados obtenidos al ejecutar las pruebas de aceptación definidas al inicio del proyecto

\section{Resultados}
\label{sec:Resultados}

Métricas del firmware desarrollado: se presentan métricas para evaluar la calidad del software desarrollado

\begin{table}[h]
	\centering
	\caption{Tabla resumen con las métricas de calidad obtenidas para el firmware desarrollado}
	\begin{tabular}{l c}    
		\toprule
		\textbf{Equipo} 	 & \textbf{Valor}\\
		\midrule
		Punku 				 & \$ 3.000\\
		\bottomrule
		\hline
	\end{tabular}
	\label{tab:MetricasFirmware}
\end{table}


Comparaciones con productos existentes: se repite la comparación de las prestaciones de los equipos existentes con el equipo anterior y con el nuevo equipo desarrollado en función de los resultados obtenidos

\begin{table}[h]
	\centering
	\caption{Cuadro comparativo del equipo actual con el anterior y con otros equipos del mercado}
	\begin{tabular}{l c}    
		\toprule
		\textbf{Equipo} 	 & \textbf{Valor}\\
		\midrule
		Punku 				 & \$ 3.000\\
		\bottomrule
		\hline
	\end{tabular}
	\label{tab:ComparacionNuevo}
\end{table}

Comparaciones de los riesgos planificados con el resultado del proyecto: se analizan los resultados obtenidos por las acciones destinadas a mitigar los riesgos identificados en la planificación del proyecto
