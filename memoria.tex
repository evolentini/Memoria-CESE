%%%%%%%%%%%%%%%%%%%%%%%%%%%%%%%%%%%%%%%%%
% Masters/Doctoral Thesis 
% LaTeX Template
% Version 2.3 (25/3/16)
%
% This template has been downloaded from:
% http://www.LaTeXTemplates.com
%
% Version 2.x major modifications by:
% Vel (vel@latextemplates.com)
%
% This template is based on a template by:
% Steve Gunn (http://users.ecs.soton.ac.uk/srg/softwaretools/document/templates/)
% Sunil Patel (http://www.sunilpatel.co.uk/thesis-template/)
%
% Template license:
% CC BY-NC-SA 3.0 (http://creativecommons.org/licenses/by-nc-sa/3.0/)
%
%%%%%%%%%%%%%%%%%%%%%%%%%%%%%%%%%%%%%%%%%

%----------------------------------------------------------------------------------------
%	PACKAGES AND OTHER DOCUMENT CONFIGURATIONS
%----------------------------------------------------------------------------------------

\documentclass[
11pt, % The default document font size, options: 10pt, 11pt, 12pt
%oneside, % Two side (alternating margins) for binding by default, uncomment to switch to one side
%chapterinoneline,% Have the chapter title next to the number in one single line
%english, % ngerman for German
spanish,
singlespacing, % Single line spacing, alternatives: onehalfspacing or doublespacing
%draft, % Uncomment to enable draft mode (no pictures, no links, overfull hboxes indicated)
%nolistspacing, % If the document is onehalfspacing or doublespacing, uncomment this to set spacing in lists to single
%liststotoc, % Uncomment to add the list of figures/tables/etc to the table of contents
%toctotoc, % Uncomment to add the main table of contents to the table of contents
parskip, % Uncomment to add space between paragraphs
%nohyperref, % Uncomment to not load the hyperref package
headsepline, % Uncomment to get a line under the header
]{MastersDoctoralThesis} % The class file specifying the document structure



\usepackage[utf8]{inputenc} % Required for inputting international characters
\usepackage[T1]{fontenc} % Output font encoding for international characters

\usepackage{palatino} % Use the Palatino font by default

\usepackage[backend=bibtex,natbib=true, style=numeric, sorting=none]{biblatex} % Use the bibtex backend for bibliography

\addbibresource{references.bib} % The filename of the bibliography

\usepackage[autostyle=true]{csquotes} % Required to generate language-dependent quotes in the bibliography

\usepackage{caption}
\usepackage{subcaption}
\usepackage{todonotes}
\usepackage{multirow}
%------------------------
\usepackage{listings}

%\usepackage[hyphens]{url}
%\usepackage[hidelinks]{hyperref}
%\hypersetup{breaklinks=true}
\urlstyle{same}
%\usepackage{cite}
\hyphenation{biblatex}
%--------------------------

\usepackage{color}

\usepackage{enumitem}
%
%----------------------------------------------------------------------------------------
%	MARGIN SETTINGS
%----------------------------------------------------------------------------------------

\geometry{
	paper=a4paper, % Change to letterpaper for US letter
	inner=2cm, % Inner margin
	outer=3.3cm, % Outer margin
	bindingoffset=2cm, % Binding offset
	top=1.5cm, % Top margin
	bottom=1.5cm, % Bottom margin
	%showframe,% show how the type block is set on the page
}

\usepackage{float}
\usepackage{placeins}

%----------------------------------------------------------------------------------------
%	ESTILO DE LOS BLOQUES DE CODIGO
%----------------------------------------------------------------------------------------

\definecolor{mygreen}{rgb}{0,0.6,0}
\definecolor{mygray}{rgb}{0.5,0.5,0.5}
\definecolor{mymauve}{rgb}{0.58,0,0.82}

\lstdefinelanguage{gherkin} {
	% list of keywords
	keywords={
		Caracter{\'i}stica,
		Caracter\'istica,
		Caracteristica,
		Como,
		Quiero,
		Para,
		Antecedentes,
		Escenario,
		Esquema de escenario,
		Ejemplos,
		Dado,
		Dada,
		Dados,
		Dadas,
		Cuando,
		Entonces,
		Y,
		Pero,
	},
	morekeywords={
		Caracter{\'i}stica,
		Caracter\'istica,
	},
	sensitive=True, % keywords are not case-sensitive
}

\lstset{ %
	backgroundcolor=\color{white},   % choose the background color; you must add \usepackage{color} or \usepackage{xcolor}
	basicstyle=\small\ttfamily,
	breakatwhitespace=false,         % sets if automatic breaks should only happen at whitespace
	breaklines=true,                 % sets automatic line breaking
	captionpos=b,                    % sets the caption-position to bottom
	commentstyle=\color{mygreen},    % comment style
	deletekeywords={...},            % if you want to delete keywords from the given language
	%escapeinside={\%*}{*)},          % if you want to add LaTeX within your code
	%extendedchars=true,              % lets you use non-ASCII characters; for 8-bits encodings only, does not work with UTF-8
	%frame=single,	                   % adds a frame around the code
	keepspaces=true,                 % keeps spaces in text, useful for keeping indentation of code (possibly needs columns=flexible)
	keywordstyle=\color{blue},       % keyword style
	language=[ANSI]C,					% the language of the code
	%otherkeywords={*,...},           % if you want to add more keywords to the set
	numbers=left,                    % where to put the line-numbers; possible values are (none, left, right)
	numbersep=5pt,                   % how far the line-numbers are from the code
	numberstyle=\tiny\color{mygray}, % the style that is used for the line-numbers
	rulecolor=\color{black},         % if not set, the frame-color may be changed on line-breaks within not-black text (e.g. comments (green here))
	showspaces=false,                % show spaces everywhere adding particular underscores; it overrides 'showstringspaces'
	showstringspaces=false,          % underline spaces within strings only
	showtabs=false,                  % show tabs within strings adding particular underscores
	stepnumber=1,                    % the step between two line-numbers. If it's 1, each line will be numbered
	stringstyle=\color{mymauve},     % string literal style
	tabsize=2,	                   % sets default tabsize to 2 spaces
	title=\lstname,                   % show the filename of files included with \lstinputlisting; also try caption instead of title
	morecomment=[s]{/*}{*/},
    inputencoding=utf8,
	extendedchars=true,
	literate=        
		{á}{{\'a}}1 {é}{{\'e}}1 {í}{{\'i}}1 {ó}{{\'o}}1 {ú}{{\'u}}1
		{Á}{{\'A}}1 {É}{{\'E}}1 {Í}{{\'I}}1 {Ó}{{\'O}}1 {Ú}{{\'U}}1
		{à}{{\`a}}1 {è}{{\`e}}1 {ì}{{\`i}}1 {ò}{{\`o}}1 {ù}{{\`u}}1
		{À}{{\`A}}1 {È}{{\'E}}1 {Ì}{{\`I}}1 {Ò}{{\`O}}1 {Ù}{{\`U}}1
		{ä}{{\"a}}1 {ë}{{\"e}}1 {ï}{{\"i}}1 {ö}{{\"o}}1 {ü}{{\"u}}1
		{Ä}{{\"A}}1 {Ë}{{\"E}}1 {Ï}{{\"I}}1 {Ö}{{\"O}}1 {Ü}{{\"U}}1
		{â}{{\^a}}1 {ê}{{\^e}}1 {î}{{\^i}}1 {ô}{{\^o}}1 {û}{{\^u}}1
		{Â}{{\^A}}1 {Ê}{{\^E}}1 {Î}{{\^I}}1 {Ô}{{\^O}}1 {Û}{{\^U}}1
		{œ}{{\oe}}1 {Œ}{{\OE}}1 {æ}{{\ae}}1 {Æ}{{\AE}}1 {ß}{{\ss}}1
		{ű}{{\H{u}}}1 {Ű}{{\H{U}}}1 {ő}{{\H{o}}}1 {Ő}{{\H{O}}}1
		{ç}{{\c c}}1 {Ç}{{\c C}}1 {ø}{{\o}}1 {å}{{\r a}}1 {Å}{{\r A}}1
		{€}{{\EUR}}1 {£}{{\pounds}}1
}

%----------------------------------------------------------------------------------------
%	INFORMACIÓN DE LA MEMORIA
%---------------------------------------------------------------------------------------

\thesistitle{Punku, control de accesos} % El títulos de la memoria, se usa en la carátula y se puede usar el cualquier lugar del documento con el comando \ttitle
\degree{Especialista en Sistemas Embebidos } % Nombre del grado, se usa en la carátula y se puede usar el cualquier lugar del documento con el comando \degreename
\author{Ing. Esteban Daniel Volentini} % Tu nombre, se usa en la carátula y se puede usar el cualquier lugar del documento con el comando \authorname
\supervisor{Ing. Juan Manuel Cruz (FIUBA, UTN-FRBA)} % El nombre del director, se usa en la carátula y se puede usar el cualquier lugar del documento con el comando \supname
\juradoUNO{Esp. Ing. Eric Pernia (FIUBA, UNQ)} % Nombre y pertenencia del un jurado se usa en la carátula y se puede usar el cualquier lugar del documento con el comando \jur1name
\juradoDOS{Esp. Ing. Diego Brengi (INTI)} % Nombre y pertenencia del un jurado se usa en la carátula y se puede usar el cualquier lugar del documento con el comando \jur2name
\juradoTRES{Esp. Ing. Alejandro Permingeat (FIUBA, USAT-Motion)} % Nombre y pertenencia del un jurado se usa en la carátula y se puede usar el cualquier lugar del documento con el comando \jur3name
\fechaINICIO{marzo de 2019}
\fechaFINAL{marzo de 2020}

\subject{Memoria del Trabajo Final de la Carrera de Especialización en Sistemas Embebidos de la UBA} % Your subject area, this is not currently used anywhere in the template, print it elsewhere with \subjectname
\keywords{Sistemas Embebidos, CESE, FIUBA} % Keywords for your thesis, this is not currently used anywhere in the template, print it elsewhere with \keywordnames
\university{Universidad de Buenos Aires} % Your university's name and URL, this is used in the title page and abstract, print it elsewhere with \univname
\faculty{{Facultad de Ingeniería}} % Your faculty's name and URL, this is used in the title page and abstract, print it elsewhere with \facname
\department{Departamento de Electrónica} % Your department's name and URL, this is used in the title page and abstract, print it elsewhere with \deptname
\group{{Laboratorio de Sistemas Embebidos}} % Your research group's name and URL, this is used in the title page, print it elsewhere with \groupname


\hypersetup{pdftitle=\ttitle} % Set the PDF's title to your title
\hypersetup{pdfauthor=\authorname} % Set the PDF's author to your name
\hypersetup{pdfkeywords=\keywordnames} % Set the PDF's keywords to your keywords


\newcaptionname{spanish}{\acknowledgementname}{Agradecimientos}
\newcaptionname{spanish}{\authorshipname}{Declaración de Autoría}
\newcaptionname{spanish}{\abbrevname}{Glosario}
\newcaptionname{spanish}{\byname}{por}

\renewcommand{\lstlistingname}{Algoritmo}% Listing -> Algorithm
\renewcommand{\lstlistlistingname}{Índice de \lstlistingname s}% List of Listings -> List of Algorithms

\renewcommand{\listtablename}{Índice de Tablas}
\renewcommand{\tablename}{Tabla} 

\addtolength{\footnotesep}{2mm} % Espacio adicional en los footnotes

\begin{document}

\frontmatter % Use roman page numbering style (i, ii, iii, iv...) for the pre-content pages

\pagestyle{plain} % Default to the plain heading style until the thesis style is called for the body content

%----------------------------------------------------------------------------------------
%	CARÁTULA
%----------------------------------------------------------------------------------------

\begin{titlepage}
	\begin{center}
		
		
		\includegraphics[width=.8\textwidth]{./Figures/logoFIUBA.png}
		\vspace{2cm}
		
		\textsc{\huge{Carrera de Especialización\\ \vspace{5px} en Sistemas Embebidos}}
		\vspace{.5cm} % Thesis type
		
		\textsc{\Large Memoria del Trabajo Final}\\[1cm] % Thesis type
		%\vspace{1.5cm}
		{\huge \bfseries \ttitle\par}\vspace{0.4cm} % Thesis title
		
		\vfill
		
		\vspace{2cm}
		\LARGE\textbf{Autor:\\
			\authorname}\\ % Author name
		
		\vspace{1.5cm}
		
		\large
		{Director:} \\
		{\supname} % Supervisor name
		
		\vspace{1cm}
		Jurados:\\	
		\jurunoname\\
		\jurdosname\\
		\jurtresname
		
		\vspace{2cm}
		\vfill
		\textit{Este trabajo fue realizado en la Ciudad de San Miguel de Tucumán, \newline entre \fechaINICIOname \hspace{1px} y \fechaFINALname.}
	\end{center}
\end{titlepage}


%----------------------------------------------------------------------------------------
%	RESUMEN - ABSTRACT 
%----------------------------------------------------------------------------------------

\begin{abstract}
\addchaptertocentry{\abstractname} % Add the abstract to the table of contents
%
%The Thesis Abstract is written here (and usually kept to just this page). The page is kept centered vertically so can expand into the blank space above the title too\ldots
\centering

El presente documento describe el desarrollo de Punku, un equipo destinado a controlar el acceso en una puerta utilizando tarjetas de proximidad. Surge por la necesidad de la empresa EQUISER de renovar un equipo existente que hoy resulta obsoleto por los avances de la tecnología. Se espera que el producto desarrollado brinde más funcionalidad con un costo de fabricación menor al del equipo actual.

El proyecto de desarrollo incluyó la especificación de requisitos, la definición de las pruebas de aceptación, el diseño del hardware contemplando la producción en escala y el desarrollo del firmware utilizando un sistema operativo de tiempo real. Para verificar el correcto funcionamiento del equipo se realizaron pruebas unitarias y de integración automatizadas.				

\end{abstract}

%----------------------------------------------------------------------------------------
%	CONTENIDO DE LA MEMORIA  - AGRADECIMIENTOS
%----------------------------------------------------------------------------------------

\begin{acknowledgements}
%\addchaptertocentry{\acknowledgementname} % Descomentando esta línea se puede agregar los agradecimientos al índice
\vspace{1.5cm}

Al todo el claustro de la EITI, y en especial a Sergio Saade, por el apoyo brindado para completar esta carrera especialización.  

\end{acknowledgements}

%----------------------------------------------------------------------------------------
%	LISTA DE CONTENIDOS/FIGURAS/TABLAS
%----------------------------------------------------------------------------------------
\renewcommand{\listtablename}{Índice de Tablas}

\tableofcontents % Prints the main table of contents

\listoffigures % Prints the list of figures

\listoftables % Prints the list of tables


%----------------------------------------------------------------------------------------
%	CONTENIDO DE LA MEMORIA  - DEDICATORIA
%----------------------------------------------------------------------------------------

\dedicatory{
	\textbf{Dedicado a Adriana, Nicolas, Lucas y Sofía}
	 
	que me apoyaron incondicionalmente en este desafío.
}  % escribir acá si se desea una dedicatoria

%----------------------------------------------------------------------------------------
%	CONTENIDO DE LA MEMORIA  - CAPÍTULOS
%----------------------------------------------------------------------------------------

\mainmatter % Begin numeric (1,2,3...) page numbering

\pagestyle{thesis} % Return the page headers back to the "thesis" style

\renewcommand{\tablename}{Tabla} 

% Incluir los capítulos como archivos separados desde la carpeta Chapters
% Descomentar las líneas a medida que se escriben los capítulos

% !TeX encoding = UTF-8
% !TeX spellcheck = es_ES
% Chapter 1

\newcolumntype{L}[1]{>{\raggedright\let\newline\\\arraybackslash\hspace{0pt}}m{#1}}
\newcolumntype{C}[1]{>{\centering\let\newline\\\arraybackslash\hspace{0pt}}m{#1}}
\newcolumntype{R}[1]{>{\raggedleft\let\newline\\\arraybackslash\hspace{0pt}}m{#1}}

\chapter{Introducción General} % Main chapter title

\label{Chapter1}
\label{IntroGeneral}

En el presente capitulo se presentan los aspectos generales del trabajo desarrollado y una breve introducción a las tecnologías utilizadas en el mismo.

\section{Descripción del problema}
\label{sec:descripcion}

El control de las personas que acceden a un ambiente es una de las aplicaciones que se benefician desde hace largo tiempo con los avances de la electrónica. El reemplazo de las cerraduras mecánicas y sus correspondientes llaves por sistemas electrónicos lleva mas de 30 años, y continua en plena expansión. Actualmente existe en el mercado una oferta de equipos muy amplia y variada, que utilizan diferente medios para identificar a las personas que intentan acceder. Los métodos de identificación mas difundidos en la actualidad son:

\begin{itemize}
	\item Clave numérica: para la identificación del usuario se le asigna una clave numérica que se ingresa mediante un teclado que forma parte del equipo. Este esquema es el mas económico pero también el más inseguro ya que la clave puede fácilmente darse a conocer a personas no autorizadas. 
	\item Tarjetas de proximidad: para la identificación se utilizan unos sistemas electrónicos que se alimentan al aproximarlos a un lector y transmiten información que permite identificar al portador. Las tarjetas en realidad pueden adoptar otros formatos como llaveros o etiquetas autoadhesivas. Este esquema resulta más costoso pero también más seguro que el anterior ya que las tarjetas no pueden replicarse fácilmente, aunque si es posible prestarlas a personas no autorizadas.
	\item Reconocimiento de huella digital: para la identificación se toma una imagen de un dedo del usuario y se reconoce la geometría de las huellas digitales del mismo. Este esquema es el mas costos y seguro de todos, ya que duplicar una huella digital es una tarea realmente compleja.
\end{itemize}

Independientemente del medio utilizado para la identificación del usuario, todos los equipos tienen una forma de operación y configuración similar, la que puede ser clasificada en tres grandes grupos:

\begin{itemize}
	\item Equipos autónomos: son los más económicos y fáciles de instalar, ya que solo requieren alimentación y la conexión con el mecanismo que libera la puerta para permitir el acceso. Dentro de este grupo podemos encontrar incluso equipos que se integran dentro de una cerradura tradicional y se alimentan con baterías, lo que simplifica al máximo la instalación de los mismos. La principal desventaja de este tipo de equipos es la gestión de las personas autorizadas a ingresar. En general estos equipos integran unos teclados muy básicos con los que se pueden agregar y borrar las personas autorizadas mediante secuencias de códigos numéricos bastante poco amigables con los usuarios finales.
	\item Equipos en linea: son los mas seguros pero también los mas costos y complejos de instalar. Estos equipos incorporan una interfaz de comunicaciones para validar cada operación de acceso con un servidor central en tiempo real. De esta forma la gestión de las personas autorizadas se realiza en forma centralizada sobre dicho servidor mediante un programa informático mucho mas simple de utilizar. Este esquema puede ademas incorporar mayor complejidad en las validaciones efectuadas para autorizar el ingreso de una persona. La principal desventaja de este esquema es que resulta muy sensible a una falla en la red de comunicaciones o en el servidor de autorizaciones.
	\item Equipos gestionados: son equipos autónomos que incorporan una interfaz de comunicaciones para permitir la gestión de los mismos en una forma más simple. En muchos casos son equipos que pueden ademas operar en linea. Podría pensarse que estos equipo combinan las desventajas de los dos anteriores, pero esto no es verdad. Si bien el costo es mayor que el de un equipo autónomo,  no requiere toda la infraestructura de los equipos que operan en linea, lo que disminuye significativamente el costo total de la solución. A cambio de ese aumento en el costo permiten una gestión mas simple ya que es posible utilizar una computadora con un software amigable.
\end{itemize}

Como se desprende del análisis anterior resulta muy difícil encontrar un equipo adecuado para el mercado hogareño o de pequeñas oficinas. A pesar de que la oferta del mercado es muy amplia no existen mucho equipos que combinen adecuadamente las características de precio con la facilidad de instalación y gestión necesarias en un ambiente donde el usuario final posee conocimientos técnicos muy limitados. Por estas razones la mayoría de los equipos destinados a este mercado corresponden al grupo de los equipos gestionados. Sin embargo la mayoría de los mismos utilizan interfaces de comunicaciones cableadas, las que complican la instalación y permiten la gestión unicamente desde una computadora.

Punku nace como una propuesta de la firma EQUISER para resolver el problema del control de accesos en hogares, pequeñas oficinas, consorcios de departamentos y cocheras. Las características mas importantes de este mercado son la poca cantidad de puertas controladas, falta de personal técnico para la gestión del equipo, y en el caso de los consorcios y cocheras, frecuentes cambios en la lista de personas autorizadas. Para lograr la mejor relación entre seguridad, precio y facilidad de gestión, el equipo puede funcionar con tarjetas de proximidad o con controles remotos. Ademas utiliza una interfaz \emph{Bluetooth} que permite gestionarlo desde un dispositivo móvil, que puede ser una computadora portátil, un teléfono celular inteligente o una tableta. También incorpora una entrada para un sensor que permite detectar la apertura de la puerta, y una salida de alarma para informar cuando la misma permanece abierta por más tiempo del adecuado. En la figura \ref{fig:EquipoActual} se puede ver una imagen del equipo actual con su correspondiente lectora de proximidad.

\begin{figure}[ht]
	\centering
	\vspace{3mm}
	\includegraphics[scale=.5]{./Figures/EquipoActual.png}
	\caption{Fotografía del equipo actual}
	\label{fig:EquipoActual}
\end{figure}

En la tabla \ref{tab:ComparacionActual} se puede ver un cuadro comparativo del producto actual con otros equipos existentes en el mercado. Como se puede observar la oferta se polariza en dos tipos de equipos: totalmente autónomos gestionados mediante un teclado numérico muy limitado o equipos con conexiones cableadas (ethernet o usb) que solo pueden ser gestionados desde computadoras de escritorio o portátiles. En el mercado internacional si existen equipos que se pueden gestionar desde un dispositivo móvil, pero tampoco en este escenario la oferta es abundante. Por estas razones Punku constituye una solución atractiva que busca imponerse principalmente por la facilidad de manejo por parte del usuario final. 

\begin{table}[ht]
	\centering
	\caption{Cuadro comparativo con otros equipos del mercado}
	\begin{tabular}{C{25mm} C{25mm} C{50mm} C{20mm}}    
		\toprule
		\textbf{Equipo}  
			& \textbf{Tecnología} 
			& \textbf{Forma de gestión}
			& \textbf{Valor}  \\
		\midrule
		EQUISER \newline Punku 
			& Proximidad y remotos RF
			& Gestionado desde un \newline celular usando bluetoth
			& \$ 3.000\\
		Tebas \newline 208NW \cite{TEBAS}
			& Proximidad
			& Teclado numérico \newline integrado en el equipo
			& \$ 4.000\\
		ZK \newline MA300IS \cite{ZK}
			& Proximidad \newline y huellas
			& Computadora conectada \newline mediante Ethernet
			& \$ 10.000\\
		ANVIZ \newline T5 Pro \cite{ANVIZ}
			& Proximidad \newline y huellas
			& Computadora conectada \newline mediante Ethernet o USB
			& \$ 8.000\\
		\bottomrule
		
		\hline
	\end{tabular}
	\label{tab:ComparacionActual}
\end{table}

\section{Motivación}
\label{sec:motivacion}

En la figura \ref{fig:BloquesActual} se puede observar un diagrama de bloques del equipo que se produce actualmente. En el mismo encontramos dos unidades funcionales: 

\begin{itemize}
	\item La lectora de proximidad: es la responsable de generar la el campo magnético que alimenta a las tarjetas y decodificar la información enviada por estas.
	\item El procesador: que es el responsable de determinar si la tarjeta leída puede o no acceder, registrar los movimientos de acceso, accionar las salidas para autorizar el ingreso, supervisar el estado de la puerta y comunicarse con el equipo móvil para permitir gestionar la configuración del mismo.
\end{itemize}

\begin{figure}[ht]
	\centering
	\includegraphics[width=\textwidth]{./Figures/BloquesActual.pdf}
	\caption{Diagrama de bloques del equipo actual}
	\label{fig:BloquesActual}
\end{figure}

La versión en producción del equipo fue diseñada en el año 2013 y los cambios de tecnologías de estos últimos seis años demandaban las siguientes actualizaciones en el diseño del mismo:

\begin{itemize}
	\item Cambio del procesador principal: para el desarrollo original se utilizó un microcontrolador de la familia \emph{ColdFire} producidos por la empresa \emph{Freescale}. Con el avance de los procesadores \emph{Cortex} el fabricante fue dejando de lado el desarrollo de esta gama de productos, por lo que en este momento el microcontrolador utilizado resulta más costoso que un procesador más nuevo con mayores prestaciones. Lamentablemente no existe una oferta por parte del fabricante de un microcontrolador con una disposición de terminales pensada para efectuar un reemplazo directo del utilizado actualmente.
	
	\item Cambio de la interfaz de comunicaciones: el equipo actual utiliza para la comunicación una interfaz \emph{Bluetoth} 2. Poco tiempo después del desarrollo se aprobó la versión 4 de este protocolo, la cual fue adoptado rápidamente por la firma \emph{Apple}. Esta nueva versión del protocolo no es compatible con la anterior, y si bien algunos equipos pueden funcionar en ambos modos de trabajo, este no es el caso de los equipos \emph{iPhone}, donde para privilegiar la duración de la batería solo pueden operar con \emph{Bluetooth} 4.
	
	\item Incorporación de un reloj: al comercializar las primeras unidades surgió el requerimiento de algunos clientes para disponer de un registro de acceso con fecha y hora de cada evento. El equipo actual no dispone de un \emph{Real Time Clock} (RTC) que le permita mantener la fecha y hora cuando el mismo no dispone de alimentación eléctrica. De esta forma si bien el equipo registra los eventos con fecha y hora, la misma vuelve a una fecha inicial preestablecida cada vez que se reinicia el mismo, y es necesaria una comunicación desde el celular para reajustar el reloj interno. El resultado es que la mayoría de los registros de accesos no tienen una fecha y hora validas.
	
	\item Soporte para cerraduras motorizadas: uno de los principales obstáculos para la instalación de equipos para el control de accesos en el mercado al cual esta destinado Punku son los cortes de alimentación eléctrica. Según el tipo de cerradura que se utilice, ante una falla de energía la puerta quedará cerrada y no se podrá acceder, o peor aun quedará abierta en forma permanente. Una solución a este problema es combinar una cerradura con llave y un sistema motorizado que permita la liberación y el bloqueo de la puerta por cualquiera de los medios. Para esto el equipo debe poder invertir la alimentación en la salida que controla al motor, y de esta forma poder liberar o bloquear el mecanismo en función del sentido de giro del mismo.
	
	\item Gestión desde la nube: con el avance en la conectividad a Internet se volvió más común disponer de equipos que pueden ser gestionados en forma remota. En el caso de un control de accesos siempre resulta atractiva la posibilidad de cambiar el permiso de acceso de una persona en forma inmediata sin necesidad de estar cerca del equipo. El uso de \emph{Bluetooth} como interfaz de comunicaciones resulta un obstáculo para esta funcionalidad por su corto alcance y porque nunca se popularizaron equipos equivalente a los \emph{routers} de \emph{WiFi} que permitan el acceso a Internet utilizando Bluetooth.
\end{itemize}

\section{Objetivos y alcance}
\label{sec:objetivos}

El objetivo general del trabajo desarrollado fue el diseño y la implementación de un prototipo de una nueva versión del equipo para control de accesos Punku que resuelva los problemas mencionados en la sección \ref{sec:motivacion}. Los objetivos particulares definidos para el mismo fueron:

\begin{enumerate}
	\item Diseñar el nuevo equipo utilizando un módulo de la familia ESP-32 del fabricante Espressif, el cual incorpora interfaces de comunicación \emph{Wifi} y \emph{Bluetooth} en las versiones 2 y 4.
	
	\item Agregar al nuevo equipo un circuito integrado RTC con respaldo de batería que le permita mantener la fecha y hora validos cuando se producen interrupciones en el suministro de energía eléctrica.
	
	\item Agregar al nuevo equipo la posibilidad de invertir la polaridad de alimentación en la salida que libera la cerradura de forma tal que se pueda utilizar un sistema motorizado para destrabar la puerta.
	
	\item Mantener las características generales del equipo actual en el nuevo diseño, tratando en lo posible de mejorar aun mas la facilidad de gestión del equipo y de disminuir el costo del mismo.
	
	\item Desarrollar el firmware del equipo desde cero, utilizando un sistema operativo de tiempo real y un diseño modular que permita escalar las funcionalidades del equipo.
\end{enumerate}

El resultado esperado es entonces un prototipo totalmente funcional del equipo, que incorpore las mejoras ya mencionadas acompañado de la siguiente documentación:

\begin{enumerate}
	\item Diagrama esquemático del equipo con en el programa KiCad.
	\item Diseño de la placa electrónica realizada en el programa KiCad.
	\item Documento con la especificación de requisitos del firmware según el estandar IEEE-830.
	\item Documento con la arquitectura y el diseño detallado del firmware modelado utilizando diagramas UML.
	\item Documento con las pruebas de aceptación para validar el correcto funcionamiento del equipo escritas en metalenguaje Gherkin. 
	\item Código fuente del firmware para el control del equipo con documentación y comentarios adecuados que faciliten la compresión del mismo.
\end{enumerate}
\chapter{Introducción Específica} % Main chapter title

\label{Chapter2}

En el presente capitulo se presentan en mayor profundidad las distintas tecnologías utilizadas en el equipo. A continuación se detallan los requisitos y casos de uso relevados, junto a las pruebas de aceptación definidas para validar el correcto funcionamiento del prototipo. Finalmente se presenta brevemente la planificación del proyecto efectuada oportunamente.

\section{Tarjetas de proximidad}
\label{sec:tarjetas}

La identificación por radiofrecuencia o \emph{Radio Frecuency Identification (RFID)} es una tecnología que permite el intercambio de información sin contacto entre un equipo lector y un dispositivo de almacenamiento denominado transpondedor. También denominadas etiquetas RFID, estos dispositivos pueden ser activos si cuentan con alimentación propia, o pasivos si se alimentan del campo electromagnético generado por el equipo lector. 

En el ámbito de las sistemas para el control de accesos la mayoría de las etiquetas RFID son pasivas y adoptan la forma de una tarjeta plástica, un llavero o una etiqueta autoadhesiva que contiene el chip electrónico junto con la bobina que cumple la función de antena receptora y transmisora. Las distancias de lectura generalmente son del orden de los 10 cm pero pueden extenderse hasta las 15 metros en los sistemas desarrollados para control de vehículos. En el mercado actual de Argentina podemos encontrar, principalmente, cuatro tipos de tarjetas de proximidad:

\begin{itemize}
	\item Tarjetas EM4100: son las más económicas y muy difundidas en el control de accesos. Desarrolladas originalmente por la empresa E\&M Marine fueron ampliamente copiadas por los fabricantes chinos. Estas tarjetas operan con una portadora de 125 KHz, la que modulan en amplitud para transmitir una trama fija de 64 bits, que transmite un número de serie prefijado en el fabricación de 24 bits.
	
	\item Tarjetas HID: son el grupo menos difundido en nuestro país, principalmente por el costo y dificultad para adquirir las mismas. Estas tarjetas operan también con una portadora de 125 KHz, pero la modulan en frecuencia. Existen tres variantes en las tramas transmitidas que  serie prefijado en el fabricación de 24 o de 36 bits según la versión de tarjeta.
	
	\item Tarjetas MIFARE: son las utilizadas por los sistemas de monederos electrónicos y pagos de pasajes en el transporte público de pasajeros. Fueron desarrolladas originalmente por la empresa Philips Semiconductors y hoy son mantenidas por NXP Semiconductors. Estas tarjetas tienen la capacidad de proteger la memoria en base a un par de claves criptográficas para impedir la lectura o escritura por parte de equipos no autorizados. Se encuentran comprendidas dentro de la norma emitida por la Organización Internacional de Normalización (ISO) para regular la operación de tarjetas de proximidad (ISO 14443).
	
	\item Tarjetas UHF: son las utilizadas en los sistema de telepeaje y sistemas para control vehiculares, porque pueden alcanzar distancias de lecturas de hasta 15 metros. Estas tarjetas también tienen la capacidad de proteger la memoria utilizando criptografía para impedir la lectura o escritura por parte de equipos no autorizados. Se encuentran normalizadas por la Organización Internacional de Normalización (ISO) para regular la operación de tarjetas de sin contacto de largo alcance (ISO 18000-6C).
\end{itemize}

En la tabla \ref{tab:TarjetasUsadas} se puede observar un cuadro comparativo entre los distintos tipos de tarjetas de proximidad utilizadas en los sistemas para control de accesos. De estas opciones de decidió que el nuevo opere con el estándar MIFARE, lo que permite utilizar todas las tarjetas de pago electrónico para el transporte publico como medio de identificación y acceso, disminuyendo de esta forma la inversión inicial de la instalación.

\begin{table}[ht]
	\centering
	\caption[Tarjetas de proximidad más utilizadas en el control de accesos]{Cuadro comparativo con las tarjetas de proximidad más utilizadas para el control de accesos}
	\begin{tabular}{C{15mm} C{40mm} C{25mm} C{35mm}}
		\toprule
		\textbf{Nombre} 	
			& \textbf{Frecuencia}
			& \textbf{Distancia}	
			& \textbf{Capacidades}
			\\
		\midrule
			EM4100 			
			& 125 KHz / ASK
			& 10 a 15 cm	
			& Solo lectura
			\\
			HID 			
			& 125 KHz / FSK
			& 10 a 15 cm	
			& Solo lectura
			\\
			MIFARE 			
			& 15.56 KHz /ASK
			& 10 a 30 cm	
			& Lectura/Escritura
			\\
			UHF 			
			& 850 a 915 MHz /ASK
			& Hasta 15 m	
			& Lectura/Escritura
			\\
		\bottomrule
		\hline
	\end{tabular}
	\label{tab:TarjetasUsadas}
\end{table}

\section{Protocolos inalámbricas}

El aumento del uso de dispositivos móviles con un poder de computo cada vez mayor por parte del publico en general abre la puerta a nuevas aplicaciones para estos equipos. En particular para el trabajo desarrollado, la intención es utilizar un dispositivo móvil como un teléfono inteligente o una tableta, para gestionar y configurar el equipo para control de accesos. Para esto es necesario que la interfaz de comunicaciones implementada sea del tipo inalámbrica, ya que si bien es técnicamente posible utilizar una interfaz USB cableada para conectarse a la mayoría de teléfono o tabletas, esta opción no resultaría cómoda ni comercialmente atractiva. En la actualidad dos protocolos de comunicación inalámbricos dominan el mercado:

\begin{itemize}
	\item Bluetooth: es un protocolo diseñado para una red de área personal, con un alcance típico de 10 metros, con bajas tasas de transferencias y optimizado para extender la duración de las baterías. Existen dos versiones principales de este protocolo: la versión 2 y la 4, las cuales no son compatibles entre si. La mayoría de los dispositivos móviles actuales soportan ambas versiones del protocolo, excepto todos los equipos móviles de la firma Apple, que solo soportan la versión 4 de bajo consumo. En este protocolo la conexión se realiza entre un maestro y un esclavo, y si bien teóricamente se pueden generar redes de dispositivos en la practica esto nunca se implementa.
	
	\item WiFi: es un protocolo diseñado para una red de área local, con un alcance típico de 100 metros y altas tasas de transferencias. Existen varias versiones de este protocolo, las cuales se pueden agrupar en dos familias: las que utilizan una portadora de 2.4 GHz y las de 5 GHz. Las versiones mas nuevas pueden utilizar ambas frecuencias simultáneamente para aumentar aun más el ancho de banda. En este protocolo la conexión se realiza generalmente entre un dispositivo y un punto de acceso, que normalmente brinda conexión con una red mayor y eventualmente con Internet.
\end{itemize}

En la tabla \ref{tab:WifiBluetooth} se puede ver un resumen con las características mas importantes de las diferentes variantes de ambos protocolos de comunicación. Para el desarrollo del equipo se decidió utilizar WiFi, de esta forma resulta igual de sencillo establecer una conexión punto a punto entre dispositivo móvil y el equipo que se quiere gestionar, como conectarlo a la red existente en el lugar y gestionarlo desde cualquier ubicación que tenga conexión con dicha red. Incluso permite en un futuro el acceso a Internet del equipo, lo que permitiría la gestión del mismo desde cualquier lugar del mundo.

\begin{table}[ht]
	\centering
	\caption[Comparación entre los protocolos Bluetooth y Wifi]{Cuadro comparativo entre las diferentes variantes de los protocolos Wifi y Bluetooth}
	\begin{tabular}{c c c c}
		\toprule
		\textbf{Nombre}	
		& \textbf{Alcance}
		& \textbf{Frecuencia}
		& \textbf{Velocidad}
		\\
		\midrule
		Bluetooth 2.0
		& 10 metros
		& 2,4 GHz
		& 2 MBits/s
		\\
		BLE
		& 10 metros
		& 2,4 GHz
		& 2 MBits/s
		\\
		Wifi 802.11a
		& 100 metros
		& 5 GHz
		& 54 MBits/s
		\\
		Wifi 802.11b
		& 100 metros
		& 2,4 GHz
		& 11 MBits/s
		\\
		Wifi 802.11g
		& 100 metros
		& 2,4 GHz
		& 54 MBits/s
		\\
		Wifi 802.11n
		& 100 metros
		& 2,4 y 5 GHz
		& 600 MBits/s
		\\
		\bottomrule
		\hline
	\end{tabular}
	\label{tab:WifiBluetooth}
\end{table}

La utilización de la interfaz WiFi implica casi inevitablemente el uso del conjunto de protocolos TPC/IP, lo que permite disponer de una serie de opciones estandarizadas para la capa de aplicación del protocolo de gestión y configuración del equipo. De estas opciones disponibles se decidió implementar una interfaz Full REST Api utilizando el protocolo HTTP. El motivo de esta elección fue simplificar la comunicación con aplicaciones híbridas para celulares, como se explica en la sección \ref{sec:AplicacionesMoviles} y paginas WEB, lo que también permitirá en un futuro la gestión del equipo desde Internet

\section{Aplicaciones para dispositivos móviles}
\label{sec:AplicacionesMoviles}

El mercado de los dispositivos móviles se encuentra polarizado en dos sistemas operativos: Android e iOS. Lamentablemente las herramientas de desarrollo e incluso los lenguajes utilizados por cada plataforma son totalmente incompatibles, esto significa que para desarrollar una aplicación que se encuentre disponible para ambas plataformas requiere el doble de esfuerzo. 

Una solución a este problema son las denominada aplicaciones híbridas, las cuales en realidad son paginas WEB encapsuladas en una con el correspondiente navegador de cada plataforma en una aplicación nativa. Estas aplicaciones se desarrollan entonces en JavaScript y uno de los entornos de desarrollo más difundidos para este tipo de aplicaciones es el conjunto Ionic Cordoba, el cual permite desarrollar una aplicación para móviles utilizando exclusivamente tecnología web. 

Dado que en realidad estas aplicaciones son paginas web tienen una serie de restricciones, principalmente en las comunicaciones que pueden realizar y en la interacción con los dispositivos del equipo móvil como la cámara o el GPS. Algunas de estas restricciones son resueltas utilizando \emph{plugins}, fragmentos de código nativo escrito para cada plataforma que funcionan a modo de adaptador para que la pagina web pueda acceder a estos servicios.

La forma más natural de comunicación para este tipo de aplicaciones es utilizando conexiones HTTP para enviar o recuperar objetos codificados según el estándar JSON. Por esta razón, y aun cuando el desarrollo de la aplicación de gestión para el dispositivo móvil esta fuera de los alcances del trabajo, se decidió implementar toda la gestión del equipo utilizando esta tecnología.

\section{Características del equipo}
\label{sec:Caracteristicas}

Dado que el objetivo del trabajo fue el diseño de un equipo comercial se planteó la necesidad de permitir que el mismo pueda adaptarse a diferentes instalaciones. La primera de las opciones de instalación corresponde al tipo de dispositivo que se utiliza para impedir la apertura de la puerta. En este apartado podemos encontrar dos opciones:

\begin{itemize}
	\item Destraba pestillo eléctrico: es un sistema mecánico que actúa como traba para el pestillo de la puerta. Este se libera al energizar una bobina y que por la acción de un resorte vuelve a bloquearse automáticamente cuando se retira la energía eléctrica. De esta forma mientras el mismo permanece energizado es posible abrir la puerta.
	
	\item Cerradura electromagnética: es simplemente un electroimán que cuando permanece alimentado ejerce una fuerza que impide la apertura de la puerta. En este caso la puerta puede abrirse únicamente cuando la cerradura permanece sin alimentación
	
	\item Cerradura motorizada: en este caso el sistema mecánico de bloqueo es accionado por un motor en lugar de por una bobina. Para liberar la puerta es necesario alimentar el motor con una determinada polaridad por un tiempo determinado, o hasta que acciona un sensor que detecta el final del recorrido del mecanismo. En este caso la puerta permanece liberada hasta que se alimenta el motor con la polaridad contraria durante un tiempo determinado, o hasta que se acciona un nuevo sensor que detecta el final del recorrido en el sentido contrario al inicial
\end{itemize}

Como se puede deducir de la descripción anterior cada tipo de cerradura requiere una señal de control diferente. En particular la mayor diferencia esta dada entre los sistemas electromecánicos y los motorizados, debido a la necesidad de una inversión en la alimentación para efectuar el bloque de la puerta. 

La otra opción de instalación que podrá definir el usuario final corresponde al uso de un sensor para detectar la apertura de la puerta. En el diseño del equipo se contempla la posibilidad de utilizar este sensor para poder determinar si una persona autorizada ingresa al espacio controlado, y para ademas informar mediante una señal de alarma cuando la puerta permanece abierta por mas tiempo del adecuado. De una análisis rápido resulta claro que el comportamiento del equipo debe ser diferente cuando el usuario decide no instalar el sensor para determinar la apertura de la puerta.  

La combinación de las dos opciones antes analizadas determina cuatro modos de funcionamiento principales, los cuales se resumen en la tabla \ref{tab:ModosOperacion}.

\begin{table}[ht]
	\centering
	\caption[Resumen de los modos de funcionamiento del equipo]{Resumen de los modos de funcionamiento del equipo en función la cerradura y la  instalación del sensor de puerta.}
	\begin{tabular}{c c L{80mm}}
		\toprule
		\textbf{Cerradura} 	& 
		\textbf{Sensor}	&
		\textbf{Forma de operación} \\
		\midrule
		Electromagnética &
		Sin instalar &
		Se libera la cerradura, cambiando el estado de la alimentación, por un tiempo predefinido. Al finalizar el mismo se bloquea la cerradura volviendo la alimentación al estado inicial. \\
		Electromagnética &
		Instalado &
		Se libera la cerradura, cambiando el estado de la alimentación, hasta detectar la apertura de la puerta sin exceder un tiempo máximo predefinido. Se supervisa que la puerta no permanezca abierta por mas de tiempo máximo \\
		Motorizada &
		Sin instalar &
		Se libera la cerradura alimentando el motor por un tiempo predefinido. Al finalizar el mismo la puerta permanece liberada por un tiempo máximo. Al terminar el mismo se bloquea la cerradura alimentando el motor con polaridad inversa durante el mismo tiempo que al inicio. \\
		Motorizada &
		Instalado &
		Se libera la cerradura alimentando el motor por un tiempo predefinido. Al finalizar se supervisa la apertura y cierre de la puerta. Al detectar la apertura de la puerta, o exceder un tiempo máximo, se bloquea la cerradura alimentando el motor con polaridad inversa durante el mismo tiempo que al inicio. \\
		\bottomrule
		\hline
	\end{tabular}
	\label{tab:ModosOperacion}
\end{table}

\subsection{Requisitos del equipo}

Uno de los primeros puntos que detalla el estándar IEE-830 para especificación de requisitos son las interfaces externas de la aplicación. Dado que el desarrollo analizado corresponde al firmware de un sistema embebido las interfaces del software tienen relación directa con las entradas y salidas de la placa electrónica. En la tabla \ref{tab:ListaInterfaces} se detallan las interfaces de entradas y salidas que debe implementar el nuevo equipo, y que por lo tanto constituyen las interfaces de firmware del mismo.

\begin{table}[ht]
	\centering
	\caption{Lista de entradas y salidas del requeridas en el equipo}
	\begin{tabular}{l C{40mm} L{65mm}}    
		\toprule
		\textbf{Nombre} & 
		\textbf{Tipo} & 
		\textbf{Descripción} \\
		\midrule
		PNK-ES001 & 
		Puerto SPI &
		Circuito integrado del lector RFID \\
		PNK-ES002 & 
		Entrada digital opto-aislada &
		Sensor de puerta abierta \\
		PNK-ES003 & 
		Salida digital con inversión de polaridad &
		Actuador de la cerradura de puerta \\
		PNK-ES004 & 
		Entrada digital sin aislación &
		Sensor de cerradura liberada \\
		PNK-ES005 & 
		Entrada digital sin aislación &
		Sensor de cerradura bloqueada \\
		PNK-ES006 & 
		Entrada digital opto-aislada &
		Pulsador para apertura manual \\
		PNK-ES007 & 
		Salida digital de contacto seco &
		Salida de alarma \\
		PNK-ES008 & 
		Salida modulada en frecuencia &
		Indicador sonoro para el usuario \\
		PNK-ES009 & 
		Salida digital con inversión de polaridad &		
		Indicador luminoso para el usuario \\
		\bottomrule
		\hline
	\end{tabular}
	\label{tab:ListaInterfaces}
\end{table}

Como se explicó en el inicio de la sección \ref{sec:Caracteristicas} uno de los requerimientos importantes del equipo es poder configurar el comportamiento del mismo para adaptarlo a diferentes tipos de instalaciones. Por esta razón se decidió identificar cada uno los parámetros de configuración requeridos para lograr la capacidad de personalización desea. La lista de parámetros definidos puede verse en la tabla \ref{tab:ListaParametros}.

\begin{table}[ht]
	\centering
	\caption[Lista de parámetros para configuración del equipo]{Lista de parámetros para configurar el comportamiento del equipo según las opciones de instalación}
	\begin{tabular}{c c L{70mm}}    
		\toprule
		\textbf{Nombre} &
		\textbf{Tipo y Rango} &
		\textbf{Descripción} \\
		\midrule
		PNK-PO001 &
		Númerico \newline100ms a 2.500ms &
		Tiempo de accionamiento del indicador luminoso PNK-ES009 cuando se produce la lectura de una tarjeta \\
		PNK-PO002 &
		100ms a 2.500ms &
		Tiempo de accionamiento del indicador sonoro PNK-ES008 cuando se concede el acceso a una tarjeta \\
		PNK-PO003 &
		100ms a 2.500ms &
		Tiempo de accionamiento del indicador sonoro PNK-ES008 cuando se deniega el acceso a una tarjeta \\
		PNK-PO004 &
		1s a 10s &
		Tiempo máximo que la puerta permanece liberara para permitir la apertura de la misma \\
		PNK-PO005 &
		100ms a 2.500ms &
		Tiempo máximo de accionamiento del motor en cerraduras motorizadas\\
		PNK-PO006 &
		1s a 60s &
		Tiempo máximo que la puerta puede permanece abierta antes de generar una señal de alarma\\
		PNK-PO007 &
		0 ó 1 &
		El sensor de puerta abierta se encuentra conectado\\
		PNK-PO008 &
		0 ó 1 &
		El sistema de liberación de la puerta requiere inversión de polaridad \\
		PNK-PO009 &
		0 ó 1 &
		El sistema de liberación de la puerta dispone de sensor para indicar el estado del mismo \\
		\bottomrule
		\hline
	\end{tabular}
	\label{tab:ListaParametros}
\end{table}



\begin{table}[ht]
	\centering
	\caption{Requisitos funcionales para el control de accesos}
	\begin{tabular}{l c}    
		\toprule
		\textbf{Nombre} 	& \textbf{Característica}	\\
		\midrule
		Elemento 			& Valor	\\
		\bottomrule
		\hline
	\end{tabular}
	\label{tab:RequisitosAcceso}
\end{table}

\begin{table}[ht]
	\centering
	\caption{Requisitos funcionales para el accionamiento de los actuadores}
	\begin{tabular}{l c}    
		\toprule
		\textbf{Nombre} 	& \textbf{Característica}	\\
		\midrule
		Elemento 			& Valor	\\
		\bottomrule
		\hline
	\end{tabular}
	\label{tab:RequisitosActuadores}
\end{table}

\begin{table}[ht]
	\centering
	\caption{Requisitos funcionales para la gestión del equipo}
	\begin{tabular}{l c}    
		\toprule
		\textbf{Nombre} 	& \textbf{Característica}	\\
		\midrule
		Elemento 			& Valor	\\
		\bottomrule
		\hline
	\end{tabular}
	\label{tab:RequisitosGestion}
\end{table}

Se detallan las restricciones impuestas por el cliente el proceso de desarrollo, materiales y características del equipo a desarrollar

\begin{table}[ht]
	\centering
	\caption{Restricciones impuestas por el cliente en el desarrollo de hardware}
	\begin{tabular}{l c}    
		\toprule
		\textbf{Nombre} 	& \textbf{Característica}	\\
		\midrule
		Elemento 			& Valor	\\
		\bottomrule
		\hline
	\end{tabular}
	\label{tab:RestriccionesHardware}
\end{table}

\begin{table}[ht]
	\centering
	\caption{Restricciones impuestas por el cliente en el desarrollo de firmware}
	\begin{tabular}{l c}    
		\toprule
		\textbf{Nombre} 	& \textbf{Característica}	\\
		\midrule
		Elemento 			& Valor	\\
		\bottomrule
		\hline
	\end{tabular}
	\label{tab:RestriccionesFirmware}
\end{table}

\subsection{Casos de uso}

Se describen los casos de uso principales del equipo

\begin{table}[ht]
	\centering
	\caption{Caso de uso Acceso por pulsador}
	\begin{tabular}{l c}    
		\toprule
		\textbf{Nombre} 	& \textbf{Característica}	\\
		\midrule
		Elemento 			& Valor	\\
		\bottomrule
		\hline
	\end{tabular}
	\label{tab:CasoPulsador}
\end{table}

\begin{table}[ht]
	\centering
	\caption{Caso de uso Acceso por tarjeta de proximidad}
	\begin{tabular}{l c}    
		\toprule
		\textbf{Nombre} 	& \textbf{Característica}	\\
		\midrule
		Elemento 			& Valor	\\
		\bottomrule
		\hline
	\end{tabular}
	\label{tab:CasoTarjeta}
\end{table}

\begin{table}[ht]
	\centering
	\caption{Caso de uso Configuración del equipo}
	\begin{tabular}{l c}    
		\toprule
		\textbf{Nombre} 	& \textbf{Característica}	\\
		\midrule
		Elemento 			& Valor	\\
		\bottomrule
		\hline
	\end{tabular}
	\label{tab:CasoConfiguracion}
\end{table}

\begin{table}[ht]
	\centering
	\caption{Caso de uso Gestión de las personas autorizadas}
	\begin{tabular}{l c}    
		\toprule
		\textbf{Nombre} 	& \textbf{Característica}	\\
		\midrule
		Elemento 			& Valor	\\
		\bottomrule
		\hline
	\end{tabular}
	\label{tab:CasoAutorizacion}
\end{table}

\begin{table}[ht]
	\centering
	\caption{Caso de uso Consulta de la bitácora de accesos}
	\begin{tabular}{l c}    
		\toprule
		\textbf{Nombre} 	& \textbf{Característica}	\\
		\midrule
		Elemento 			& Valor	\\
		\bottomrule
		\hline
	\end{tabular}
	\label{tab:CasoBitacora}
\end{table}

\subsection{Pruebas de aceptación del equipo}

Se detallan las pruebas a ejecutar una vez terminado el equipo para considerar el desarrollo del mismo como completo

\section{Planificación}

Análisis inicial del proyecto

Se listan las tareas que se deben desarrollar para completar el proyecto con una breve descripción para cada una

\begin{table}[ht]
	\centering
	\caption{Lista del desglose de tareas del proyecto}
	\begin{tabular}{l c}    
		\toprule
		\textbf{Nombre} 	& \textbf{Característica}	\\
		\midrule
		Elemento 			& Valor	\\
		\bottomrule
		\hline
	\end{tabular}
	\label{tab:ListaTareas}
\end{table}

Se listan los recursos necesarios y los periodos de tiempo en los cuales los mismos deberán estar disponibles para el desarrollo del equipo

\begin{table}[ht]
	\centering
	\caption{Lista de recursos requeridos por el proyecto}
	\begin{tabular}{l c}    
		\toprule
		\textbf{Nombre} 	& \textbf{Característica}	\\
		\midrule
		Elemento 			& Valor	\\
		\bottomrule
		\hline
	\end{tabular}
	\label{tab:ListaRecursos}
\end{table}

Se presenta el análisis de riesgos efectuado al iniciar el proyecto

\begin{table}[ht]
	\centering
	\caption{Tabla de riesgos mitigados del proyecto}
	\begin{tabular}{l c}    
		\toprule
		\textbf{Nombre} 	& \textbf{Característica}	\\
		\midrule
		Elemento 			& Valor	\\
		\bottomrule
		\hline
	\end{tabular}
	\label{tab:ListaRiesgos}
\end{table}

Se presenta la planificación realizada al iniciar el desarrollo del equipo

\begin{figure}[ht]
	\centering
	%	\includegraphics[scale=.3]{./Figures/cuadradoAzul.png}
	\caption{Fotografía del equipo actual}
	\label{fig:DiagramaAON}
\end{figure}

\begin{figure}[ht]
	\centering
	%	\includegraphics[scale=.3]{./Figures/cuadradoAzul.png}
	\caption{Fotografía del equipo actual}
	\label{fig:DiagramaGantt}
\end{figure}


% ---------------------------------- Aviso fin de avance ------------------------------------
\begin{center}
	{\Large\color{red} De aquí en adelante solo esta la estructura del documento}
\end{center}
% -------------------------------------------------------------------------------------------
 
\chapter{Diseño e Implementación} % Main chapter title
\label{Chapter3} 

Párrafo introductorio.

\section{Diseño del hardware}
\label{sec:hardware}

Bloques constructivos del hardware, se presenta el diagrama de bloques y se describen los mismos

\begin{figure}[ht]
	\centering
	%	\includegraphics[scale=.3]{./Figures/cuadradoAzul.png}
	\caption{Diagrama de bloques del equipo desarrollado}
	\label{fig:DiagramaBloques}
\end{figure}

Selección de los componentes, se presentan los criterios utilizados para en la selección de los componentes.

\section{Prototipo del hardware}
\label{sec:prototipo}

Diseño y construcción de la placa electrónica, se presentan los criterios y el proceso utilizado para el diseño y construcción de la placa electrónica del prototipo.

Montaje del prototipo, se detallan los problemas de construcciones del primer prototipo y se mencionan las correcciones en el diseño de la placa electrónica efectuadas a partir de los problemas de montaje de los componentes.

\begin{figure}[ht]
	\centering
	%	\includegraphics[scale=.3]{./Figures/cuadradoAzul.png}
	\caption{Errores encontrados durante el montaje del primer prototipo}
	\label{fig:ErroresMontaje}
\end{figure}

\section{Diseño del firmware}
\label{sec:firmware}

\subsection{Arquitectura del firmware}

Se presenta la arquitectura seleccionara para el firmware del equipo, se presenta el diagrama de componentes de software del mismo y se describen brevemente las capas del mismo

\begin{figure}[ht]
	\centering
	%	\includegraphics[scale=.3]{./Figures/cuadradoAzul.png}
	\caption{Diagrama de componentes del firmware del equipo}
	\label{fig:DiagramaComponentes}
\end{figure}

\begin{figure}[ht]
	\centering
	%	\includegraphics[scale=.3]{./Figures/cuadradoAzul.png}
	\caption{Diagrama de clases del firmware del equipo}
	\label{fig:DiagramaClases}
\end{figure}

\subsection{Capa de abstracción del hardware}

Se describen las clases que componen la capa de abstracción de hardware

\subsection{Capa de controladores}

Se describen las clases que componen la capa de controladores

\begin{figure}[ht]
	\centering
	%	\includegraphics[scale=.3]{./Figures/cuadradoAzul.png}
	\caption{Diagrama de estado para el control de una puerta, sin sensor de apertura y con liberación electromagnética}
	\label{fig:ControlSinSin}
\end{figure}

\begin{figure}[ht]
	\centering
	%	\includegraphics[scale=.3]{./Figures/cuadradoAzul.png}
	\caption{Diagrama de estado para el control de una puerta, con sensor de apertura y con liberación electromagnética}
	\label{fig:ControlSinCon}
\end{figure}

\begin{figure}[ht]
	\centering
	%	\includegraphics[scale=.3]{./Figures/cuadradoAzul.png}
	\caption{Diagrama de estado para el control de una puerta, sin sensor de apertura y con liberación motorizada}
	\label{fig:ControlConSin}
\end{figure}

\begin{figure}[ht]
	\centering
	%	\includegraphics[scale=.3]{./Figures/cuadradoAzul.png}
	\caption{Diagrama de estado para el control de una puerta, con sensor de apertura y con liberación motorizada}
	\label{fig:ControlConCon}
\end{figure}

\subsection{Tareas del sistema}

Se describen las tareas del sistema operativo de tiempo real y las interacciones entre las mismas.

\begin{figure}[ht]
	\centering
	%	\includegraphics[scale=.3]{./Figures/cuadradoAzul.png}
	\caption{Diagrama de secuencia para la liberación por pulsador de una puerta, sin sensor de apertura y con liberación electromagnética}
	\label{fig:SecuanciaPulsador}
\end{figure}

\begin{figure}[ht]
	\centering
	%	\includegraphics[scale=.3]{./Figures/cuadradoAzul.png}
	\caption{Diagrama de secuencia para la apertura y cierre por tarjeta de proximidad, con sensor de apertura y con liberación electromagnética}
	\label{fig:SecuenciaTarjeta}
\end{figure}

\begin{figure}[ht]
	\centering
	%	\includegraphics[scale=.3]{./Figures/cuadradoAzul.png}
	\caption{Diagrama de secuencia para la activación de alarma por apertura forzada de la puerta}
	\label{fig:SecuenciaForzada}
\end{figure}

\begin{figure}[ht]
	\centering
	%	\includegraphics[scale=.3]{./Figures/cuadradoAzul.png}
	\caption{Diagrama de secuencia para el cambio de configuración del equipo}
	\label{fig:SecuenciaConfiguracion}
\end{figure}

\section{Desarrollo del firmware}
\label{sec:desarrollo}

\subsection{Entorno de desarrollo}

Se describen las herramientas utilizadas para el desarrollo del software

\begin{table}[ht]
	\centering
	\caption{Lista de las herramientas utilizadas para el desarrollo del firmware}
	\begin{tabular}{l c}    
		\toprule
		\textbf{Nombre} 	& \textbf{Característica}	\\
		\midrule
		Elemento 			& Valor	\\
		\bottomrule
		\hline
	\end{tabular}
	\label{tab:HerramientasDesarrollo}
\end{table}

\subsection{Uso del tipo de abstracto de datos}

Se presenta el patrón de programación denominado tipo abstracto de datos y se explica su uso en la implementación del firmware del equipo

\subsection{Pruebas de integración en las tareas del sistema}

Se presenta la problemática de implementar pruebas automatizadas en tareas del sistema operativo FreeRTOS y se explica la herramienta desarrollada para resolver este problema

% Chapter Template

\chapter{Ensayos y Resultados} % Main chapter title
\label{Chapter4}

En este capítulo se detallan las pruebas efectuadas sobre el hardware y el firmware a lo largo del desarrollo y se analizan los resultados obtenidos.
 
\section{Pruebas de concepto}
\label{sec:PruebasConcepto}

Para determinar la factibilidad de los objetivos planteados en la sección \ref{sec:objetivos} y poder, al mismo tiempo, cumplir con los requerimientos funcionales detallados en la sección \ref{sub:Requisitos}, se implementó en la primera etapa del proyecto un modelo destinado validar todas las suposiciones efectuadas en el diseño preliminar. Para esto se utilizó una placa de evaluación ESP32-DEVKITC-32D-F ofrecida por la firma Expressif, pensada para facilitar este tipo de procesos. Usando esta placa como núcleo, se construyó en forma cableada una maqueta de lo que sería el equipo definitivo. Una versión de este montaje puede observase en la figura \ref{fig:Demostrador}.

\begin{figure}[ht]
	\centering
	\includegraphics[width=0.8\textwidth]{Figures/Demostrador.png}
	\caption[Montaje utilizado para las pruebas de concepto]{Imagen con el montaje de componentes en torno a la placa de evaluación utilizado para la prueba de concepto.}
	\label{fig:Demostrador}
\end{figure}

De forma similar, utilizando los ejemplos de software provisto por el fabricante en el ESP-IDF, se empezaron a estudiar y probar las bibliotecas de software que se utilizarían en el desarrollo. Así se pudo validar cada uno de los bloques del código, para empezar a integrarlos en lo que sería la primera versión del firmware definitivo. 

Como resultado de este proceso se obtuvo una primera versión del equipo que podía leer una tarjeta de proximidad, determinar la fecha y hora mediante un RTC, escribir en una tarjeta de memoria microSD el identificador de la etiqueta RFID leída junto con la fecha y hora del evento, y transmitir esta información mediante WiFi a una computadora.

\section{Pruebas funcionales del hardware}
\label{sec:pruebasHW}

La fabricación del prototipo de hardware se dividió en dos etapas: primero se montaron todos los componentes correspondientes a la etapa de alimentación del equipo para verificar que todas las tensiones generadas fueran correctas. A continuación se montaron el resto de los componentes, y se verificó, en forma ordenada, el correcto funcionamiento de cada uno utilizando las siguientes pruebas:

\begin{enumerate}
	\item Para verificar el funcionamiento del procesador se lo configuró para iniciar en modo \emph{bootloader} y grabarle un programa de prueba. De esta forma se determinó que el procesador no completaba el proceso de \emph{reset} porque faltaba una resistencia de \emph{pull-up} en el terminal de \emph{Enable}. Para resolver este problema se agregó la resistencia R27, que se puede ver en la figura \ref{fig:ErroresPrototipo}.
	
	\item Continuando con la verificación del procesador, se lo configuró para iniciar en modo normal y ejecutar el programa cargado en el apartado anterior. Durante esta prueba se determinó que el procesador no podía iniciar en modo normal porque faltaba una resistencia de \emph{pull-up} en el terminal que determina el modo de arranque. Para resolver este problema se agregó la resistencia R28, que se puede ver en la figura \ref{fig:ErroresPrototipo}.
	
\begin{figure}[ht]
	\centering
	\includegraphics[width=0.9\textwidth]{Figures/LadoSoldadura.png}
	\caption[Errores detectados durante el montaje del prototipo]{Imagen con las correcciones efectuadas en el prototipo para resolver los errores detectados durante el montaje.}
	\label{fig:ErroresPrototipo}
\end{figure}

	\item Con el procesador en funcionamiento, se probaron las entradas y salidas digitales con programas simples que activan una salida como respuesta a un cambio en una entrada. Para facilitar la identificación de las causas de los problemas, todos estos programas de prueba utilizaron las funciones de consola disponibles en este procesado. De esta forma se pueden ver en una consola serial de la computadora de desarrollo mensajes generados en el programa del procesador embebido utilizando el mismo puerto de comunicaciones por el que se realizan los cambios de firmware.
	
	\item Para verificar el funcionamiento de la tarjeta microSD y el sistema de archivos se utilizó un programa de prueba simple para crear un archivo y almacenar en él una cadena de texto. Para certificar que funcionamiento fuese correcto se retiró la tarjeta de memoria del equipo y se comprobó el contenido del archivo en una computadora.
	
	\item Para verificar la comunicación con el circuito integrado responsable de la lectura de las tarjetas de proximidad se utilizó directamente la maqueta del firmware desarrollada en las pruebas de concepto. Tampoco se encontraron inconvenientes en la comunicación con la lectora de proximidad en el prototipo de hardware.
	
	\item Para verificar el funcionamiento del RTC también se recurrió a la maqueta del firmware utilizada en las pruebas del apartado anterior. En este caso se encontró un error importante en las conexiones del circuito integrado, originado por un error en la numeración de los terminales al definir el componente en Kicad, el programa donde se realizó el diseño de toda la placa electrónica. Este error se puede apreciar en la figura \ref{fig:ErrorRTC}, al comparar la información correspondiente a la hoja de datos, en la mitad izquierda de la figura, con la definición del componente, ubicada en la mitad derecha, vemos que todos los terminales del lado derecho se encuentran invertidos. En la figura \label{fig:ErroresPrototipo} se pueden observar las correcciones efectuadas sobre el prototipo para subsanar este problema.
	
\begin{figure}[ht]
	\centering
	\includegraphics[width=0.9\textwidth]{Figures/ErrorHuella.png}
	\caption[Error en la asignacion de terminales del RTC]{Imagen que compara la asignación de terminales del RTC según la hoja de datos, a la izquierda, con la efectuada en la huella del componente Kicad.}
	\label{fig:ErrorRTC}
\end{figure}

	\item Finalmente, a lo largo de los sucesivos reinicios de la placa se determinó que al conectar la alimentación del equipo, el procesador inciaba en modo \emph{bootloader} en lugar de ejecutar el programa ya grabado. Se comprobó también que esto no sucedía al efectuar un \emph{reset} sin interrumpir la alimentación del equipo. Se determinó al capacitor C4 como el responsable y se removió del diseño, como se puede observar en la figura \label{fig:ErroresPrototipo}.
\end{enumerate}

\FloatBarrier

\section{Pruebas unitarias y de integración}
\label{sec:PruebasFirmware}

Como ya se explicó en la sección \ref{sec:desarrollo} todo el desarrollo del firmware se planteó utilizando la metodología TDD. Esto implica que se escribieron pruebas unitarias y de integración para la mayoría de los componentes de software desarrollados. Una forma cuantitativa de evaluar estas pruebas son los informes de cobertura generados Ceedling, la herramienta utilizada para ejecutar las pruebas automatizadas y consolidar los resultados. En la figura \ref{fig:Cobertura} se puede observar el informe de cobertura, donde se puede apreciar que las pruebas automatizadas ejecutan casi del 95\% de las líneas de código escritas y explora más del 80\% de las combinaciones en los saltos condicionales.

\begin{figure}[ht]
	\centering
	\includegraphics[width=\textwidth]{Figures/Cobertura.png}
	\caption[Informe con la cobertura de las pruebas]{Imagen con el informe de cobertura de las pruebas automatizadas efectuadas sobre el firmware.}
	\label{fig:Cobertura}
\end{figure}

Para dimensionar el esfuerzo realizado en la documentación y pruebas del código se utilizó la herramienta CLOC. Esta permite contabilizar las líneas de código fuente y las de comentarios para cada uno de los archivos. Aplicándola en forma separada para el código de producción y las pruebas automatizadas se obtuvieron los valores que se muestran en la tabla \ref{tab:MetricasFirmware}. Estos se pueden resumir en la siguiente afirmación: por cada diez líneas de código de producción se escribieron siete de documentación y cuatro de pruebas automatizadas.

\begin{table}[h]
	\centering
	\caption[Metricas de documentación y pruebas]{Tabla resumen con las métricas de documentación y pruebas obtenidas para el firmware desarrollado.}
	\begin{tabular}{l c c c c c }
		\toprule
		\textbf{Grupo} &
		\textbf{Tipo} &
		\textbf{Archivos} &
		\textbf{Código} &
		\textbf{Comentarios} &
		\textbf{Relación} \\
		\midrule
		\multirow[c]{2}{*}{Producción}
				& Cabeceras & 25 & 1078 & 2419 & 2,24 \\
				& Código C 	& 21 & 3770 & 1243 & 0,33 \\
		Pruebas & Código C 	& 11 & 1813 & 1010 & 0,56 \\
		\textbf{Totales} &		    & 	 & \textbf{6661} & \textbf{4672} & 0,\textbf{70}  \\
		\bottomrule
		\hline
	\end{tabular}
	\label{tab:MetricasFirmware}
\end{table}

\section{Resultado de las pruebas}
\label{sec:PruebasFirmware}

Los resultados de las pruebas efectuadas sobre la placa electrónica, despues de efectuar las correcciones mencionadas en las secciones \ref{sec:hardware} y \ref{sec:pruebasHW}, muestran un buen comportamiento del diseño, el cual se mostró estable durante todas las pruebas del firmware. En lo que respecta al firmware, por haber sido desarrollado utilizando la metodología TDD, el mismo cumplió con el comportamiento esperado sin necesidad de efectuar cambios. 

Es importante aclarar que las pruebas de aceptación del equipo no se ejecutaron en forma automática, ya que para ello es necesario el desarrollo de hardware específico que genere los estímulos y verifique las reacciones de la placa bajo prueba. Por esta razón estas pruebas se ejecutaron manualmente, siguiendo los guiones escritos en Gherkin que se mostraron en la sección \ref{sub:PruebasAceptacion}, y todas fueron completadas en forma exitosa por el equipo desarrollado. 





 
\chapter{Conclusiones}
\label{Chapter5}

\section{Resultados Obtenidos}

Se presentan brevemente los objetivos del proyecto y las características esperadas del equipo y se contrastan con los resultados obtenidos

\section{Trabajo Futuro}

Se presentan las ampliaciones en la funcionalidad del equipo en un futuro cercano 

%----------------------------------------------------------------------------------------
%	CONTENIDO DE LA MEMORIA  - APÉNDICES
%----------------------------------------------------------------------------------------

\appendix % indicativo para indicarle a LaTeX los siguientes "capítulos" son apéndices

% Incluir los apéndices de la memoria como archivos separadas desde la carpeta Appendices
% Descomentar las líneas a medida que se escriben los apéndices

%\include{Appendices/AppendixA}
%\include{Appendices/AppendixB}
%\include{Appendices/AppendixC}

%----------------------------------------------------------------------------------------
%	BIBLIOGRAPHY
%----------------------------------------------------------------------------------------

\Urlmuskip=0mu plus 1mu\relax
\raggedright
\printbibliography[heading=bibintoc]

%----------------------------------------------------------------------------------------

\end{document}  
